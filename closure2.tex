\documentclass[11pt,a4paper]{article}
\usepackage{fullpage}
\usepackage[utf8]{inputenc}
\usepackage[english]{babel}
\usepackage{amsmath, amsthm}
\usepackage{amsfonts}
\usepackage{amssymb}
\usepackage{graphicx}
%\usepackage{bussproofs}
\usepackage{enumitem}
\usepackage{centernot}
\usepackage{mathpartir}
\usepackage{xspace}
\usepackage{stmaryrd}
\usepackage{mathtools}
\usepackage{listings}
\usepackage{tikz-qtree}
\usepackage{tikz}
\usetikzlibrary{automata,trees,fit,backgrounds,shapes,snakes}
\usetikzlibrary{decorations.shapes}

\author{Lee Gao (lg342)}
\title{Closure Conversion by explicit Lambda Abstraction}
\date{\today}

\definecolor{light-gray}{gray}{0.95}
\newcommand {\conf} [1] {\ensuremath{\left\langle #1 \right\rangle}}
\newcommand {\bstep} {\ensuremath{\Downarrow}}
\newcommand {\bstepA} {\bstep_{ A}}
\newcommand {\bstepB} {\bstep_{ B}}
\newcommand {\bstepC} {\bstep_{ C}}
%\newcommand {\co} [1] {\ensuremath{\operatorname{\bf #1}}}
\newcommand {\coo} [1] {\ensuremath{\operatorname{\mathsf{#1}}}}
\newcommand {\co} [1] {\coo{#1}}
\newcommand {\pp}  {\ensuremath{\mbox{\footnotesize{++}}}}
\newcommand {\Skip} {\co{skip}}
\newcommand {\Not} {\co{not}}
\newcommand {\If}[3] {\co{if} (#1) \co{then} #2 \co{else} #3}
\newcommand {\While}[2] {\co{while} #1 \co{do} #2}
\newcommand {\Repeat}[2] {\co{repeat} #1 \co{do} #2}
\newcommand {\Input} {\co{input}}
\newcommand {\Break} {\co{break}}
\newcommand {\Continue} {\co{continue}}
\newcommand{\True}{\co{True}}
\newcommand{\False}{\co{False}}
\newcommand{\Or}{\co{or}}
\newcommand{\Let}[1]{\coo{let} #1 \coo{in} }
\newcommand{\Lam}{\ensuremath{{\lambda}}}
\newcommand{\Ref}{\coo{ref}}
\newcommand{\Int}{\coo{int}}
\newcommand{\bool}{\coo{bool}}
\newcommand{\Bool}{\bool}
\newcommand{\Unit}{\coo{unit}}
\newcommand{\Rec}[1]{\left\{#1\right\}}
\newcommand{\pa}[1]{\left(#1\right)}
\newcommand{\ba}[1]{\left\langle #1\right\rangle}
\newcommand{\tree}{\coo{tree}}
\newcommand{\fa}{\coo{\forall}}
\newcommand{\more}[1]{\vdots\hspace{-1mm}~^{#1}}
\newcommand{\f}[1]{\textsc{#1}}
\newcommand{\g}[1]{\textsf{#1}}
\newcommand{\finite}{\co{finite}}
\newcommand{\lift}[1]{\left\lfloor #1 \right\rfloor}

\newcommand{\trans}[2]{\ensuremath{\mathcal{#1}\llbracket #2\rrbracket}}

\newtheorem*{lemma}{Lemma}
\newtheorem*{theorem}{Theorem}
\newtheorem*{definition}{Definition}

\lstset{ %
  language=Caml,                % the language of the code
  basicstyle=\footnotesize,           % the size of the fonts that are used for the code
  numbers=left,                   % where to put the line-numbers
  numberstyle=\tiny\color{gray},  % the style that is used for the line-numbers
  stepnumber=1,                   % the step between two line-numbers. If it's 1, each line 
                                  % will be numbered
  numbersep=10pt,                  % how far the line-numbers are from the code
  backgroundcolor=\color{white},      % choose the background color. You must add \usepackage{color}
  showspaces=false,               % show spaces adding particular underscores
  showstringspaces=false,         % underline spaces within strings
  showtabs=false,                 % show tabs within strings adding particular underscores
  mathescape=true,
  frame=leftline,                   % adds a frame around the code
  rulecolor=\color{gray},        % if not set, the frame-color may be changed on line-breaks within not-black text (e.g. comments (green here))
  tabsize=2,                      % sets default tabsize to 2 spaces
  captionpos=t,                   % sets the caption-position to bottom
  breaklines=true,                % sets automatic line breaking
  breakatwhitespace=false,        % sets if automatic breaks should only happen at whitespace
  title=\lstname,                   % show the filename of files included with \lstinputlisting;
                                  % also try caption instead of title
  escapeinside={\%*}{*)},            % if you want to add LaTeX within your code
  morekeywords={*,...},              % if you want to add more keywords to the set
  deletekeywords={...}              % if you want to delete keywords from the given language
}

\newcommand\tmark[2]{%
  \ensuremath{\tikz[baseline] \node[anchor=base] (#1) {#2};}}
\tikzstyle{every picture}+=[remember picture]
\newcommand{\tm}[2]{\tmark{#1}{\ensuremath{#2}}}
\newcommand{\tr}[2]{\tmark{#1}{\color{red}{\ensuremath{#2}}}}
\newcommand{\arr}[1]{\begin{array}{cccccccccc} #1\end{array}}
\newcommand{\mat}[1]{\left(\arr{#1}\right)}

\allowdisplaybreaks

\begin{document}
\tikzset{every tree node/.style={minimum width=2em,draw,circle},
         blank/.style={draw=none},
         edge from parent/.style=
         {draw, edge from parent path={(\tikzparentnode) -- (\tikzchildnode)}},
         level distance=1.5cm}
\maketitle
\setlength{\parindent}{0pt}

We want to translate from the nonclosed \f{il1} language
\begin{align*}
v &::= n \mid x \mid \lambda xk.c \mid \g{halt} \\
e &::= v \mid v_0 + v_1 \mid (v_i) \mid \pi_n v \\
c &::= \Let{x = e}{c} \mid v_0~v_1~v_2
\end{align*}
into the closed restriction with an explicit abstraction $\bar\rho \in \g{var}~\g{list}$ over all of the variables.
\begin{align*}
v &::= n \mid x \mid \lambda xk\bar\rho.c \mid \g{halt} \\
e &::= v \mid v_0 + v_1 \mid (v_i) \mid \pi_n v \\
c &::= \Let{x = e}{c} \mid v_0~v_1~v_2~\bar\rho
\end{align*}

Here, we can proceed as natural with the translation functions
\begin{align*}
\trans{V}{v} &: v \\
\trans{E}{e} &: e \\
\trans{C}{c} &: c 
\end{align*}
\section{Values}
\begin{align*}
\trans{V}{n} &= n \\
\trans{V}{x} &= \rho_{\ba{x}} \\
\trans{V}{\g{halt}} &= \g{halt} \\
\trans{V}{\lambda xk.c} &= \lambda xk\bar\rho. \Let{\rho_{\ba{x}} = x, \rho_{\ba{k}} = k}{\trans{C}{c}}
\end{align*}
\section{Expressions}
\begin{align*}
\trans{E}{v} &= \trans{V}{v} \\
\trans{E}{v_0 + v_1} &= \trans{V}{v_0} + \trans{V}{v_1} \\
\trans{E}{(v_i)} &= (\trans{V}{v_i}) \\
\trans{E}{\pi_n v} &= \pi_n \trans{V}{v}
\end{align*}
\section{Commands}
\begin{align*}
\trans{C}{v_0~v_1~v_2} &= \trans{V}{v_0} ~ \trans{V}{v_1} ~ \trans{V}{v_2} ~ \bar\rho \\
\trans{C}{\Let{x = e}{c}} &= \Let{\rho_{\ba{x}} = \trans{E}{e}}{\trans{C}{c}}
\end{align*}

For example, suppose I have the command 
$$
c \triangleq \Let{f = \lambda xk. k~ (x + 1)~k}{f~4~\g{halt}}
$$
there are three variables here, $x,k,f$, so we can construct the injection:
$$
\ba{\cdot} = \Rec{x \mapsto 1, k \mapsto 2, f \mapsto 3}
$$
so
$$
\bar\rho = \rho_1, \rho_2, \rho_3
$$
then the above would translate into
\begin{align*}
\trans{C}{c} &= \Let{\rho_{\ba{f}} = \trans{E}{\lambda xk. k (x + 1)k}}{\trans{C}{f~4~\g{halt}}} \\
\intertext{expand the $\mathcal{E}$ first}
&= \Let{\rho_{\ba{f}} = \trans{V}{\lambda xk. k (x + 1)k}}{\trans{C}{f~4~\g{halt}}} \\
&= \Let{\rho_{\ba{f}} = \lambda xk\rho_1\rho_2\rho_3.\Let{\rho_{\ba{x}} = x,\rho_{\ba{k}} = k}{\trans{C}{k~(x + 1)~k}}}{\trans{C}{f~4~\g{halt}}} \\
\vdots \\
&= \Let{\rho_{\ba{f}} = \lambda xk\rho_1\rho_2\rho_3.\Let{\rho_{\ba{x}} = x,\rho_{\ba{k}} = k}{\rho_{\ba{k}}(\rho_{\ba{x}} + 1)\rho_{\ba{k}}\rho_1\rho_2\rho_3}}{{\rho_{\ba{f}}~4~\g{halt}~\rho_1\rho_2\rho_3}} \\
&=\Let{\rho_{3} = \lambda xk\rho_1\rho_2\rho_3.\Let{\rho_{1} = x,\rho_{2} = k}{\rho_{2}(\rho_{1} + 1)\rho_{2}\rho_1\rho_2\rho_3}}{{\rho_{3}~4~\g{halt}~\rho_1\rho_2\rho_3}}
\end{align*}
\end{document}
